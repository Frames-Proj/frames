%%
%% first.text
%%

\documentclass[twocolumn, 12pt]{article}

\title{SAFEnet Project Proposal}
\author{
  Ethan Pailes
  \and
  Graham Goudeau
  \and
  Yueming Luo
  \and
  Abdisalan Mohamud
  \and
  Nicholas Yan
}
\date{}

\begin{document}

\maketitle

\section{Team}
Our project team includes Ethan Pailes, Graham Goudeau, Yueming Luo,
Abdisalan Mohamud, and Nicholas Yan.

\section{Sponsorship}

Brett Russell, James Bach, Nicholas Koteskey, and Stephan Pilsl,
members of the SAFEnet community have agreed to sponsor our project.
They are all interested and enthusiastic about the implications of
a distributed peer-to-peer network like the SAFEnet, and are willing
to provide their time in order to help grow the SAFE ecosystem.

Our sponsors are mostly located near the Prime Meridian, so we will
have to meet and coordinate online. We plan on using video chat services
such as Google Hangouts and Skype to hold these remote meetings, and we
will keep a written record of our activity to ensure that our sponsors
have enough visibility into the project.

We currently have an online meeting planned to discuss our thoughts on
the implications of the SAFEnet and get the input of our sponsors. Networks
like the SAFEnet are very new, so much of the value our sponsors will bring
to the project will be in helping us refine our ideas about it. All of them
have been actively involved in the SAFE community for some time, and have
had ample opportunity to think deeply about the network.

\section{Goals}

The SAFE (Secure Access For Everyone) network, dubbed the SAFEnet,
is a network of unused space and processing power of those on the network.
It creates a level of security for each user that exceeds any system
currently on the internet due to its distributed and encrypted data
storage protocol. Since private data is distributed -- spaced
across many hard drives on the network -- the user is in complete
control of their data.

The SAFEnet is still in its infancy though, and we're interested in
contributing to its growth and prosperity. We seek to approach this
project with two goals:

One, to create an interface for developers to use the network as a
graph based database. This will allow for easy management of
public user data as well as well as facilitate developing applications
on the SAFEnet.

Two, to create an application on the SAFEnet leveraging the database.

\section{Work Areas}

The SAFE Network is poised to make a big difference in how developers
write safe, reliable applications.  It is, however, still in its alpha
stage of development.  This means that there is not much documentation
for the software that already exists, and what does exist is relatively
sparse.  Furthermore, there are not many developers in the world who
have an active base of knowledge about the SAFE Network.

Our project would help to bring a cutting-edge piece of technology closer
to maturity.  We would do this by learning the nuances of working with
SAFEnet and contributing to the community's documentation, effectively
paving the way for future SAFEnet developers.  Another way that our
project would contribute to the growth of SAFEnet is by demonstrating a
valid use case for SAFEnet along with a workable solution to that problem.
This solution would involve a data-schema library that allows users to
specify the graph of their data across the network in a declarative manner
rather than having to go through the process manually.  We would then use
this library to build an app to demonstrate the capabilities of the network.

\end{document}
